%%%%%%%%%%%%%%%%%%%%%%%%%%%%%%%%%%%%%%%%%
% Plasmati Graduate CV
% LaTeX Template
% Version 1.0 (24/3/13)
%
% This template has been downloaded from:
% http://www.LaTeXTemplates.com
%
% Original author:
% Alessandro Plasmati (alessandro.plasmati@gmail.com)
%
% License:
% CC BY-NC-SA 3.0 (http://creativecommons.org/licenses/by-nc-sa/3.0/)
%
% Important note:
% This template needs to be compiled with XeLaTeX.
% The main document font is called Fontin and can be downloaded for free
% from here: http://www.exljbris.com/fontin.html
%
%%%%%%%%%%%%%%%%%%%%%%%%%%%%%%%%%%%%%%%%%

%----------------------------------------------------------------------------------------
%	PACKAGES AND OTHER DOCUMENT CONFIGURATIONS
%----------------------------------------------------------------------------------------

\documentclass[a4paper,11pt]{extarticle} % Default font size and paper size

\usepackage{fontspec} % For loading fonts
\defaultfontfeatures{Mapping=tex-text}
\setmainfont {AGaramondPro-Regular.otf}[
BoldFont = AGaramondPro-Bold.otf ,
ItalicFont = AGaramondPro-Italic.otf ,
BoldItalicFont = AGaramondPro-BoldItalic.otf ] % Main document font

\usepackage{xunicode,xltxtra,url,parskip} % Formatting packages

\usepackage[usenames,dvipsnames]{xcolor} % Required for specifying custom colors

%\usepackage[big]{layaureo} % Margin formatting of the A4 page, an alternative to layaureo can be 
%\usepackage{fullpage}
\usepackage{geometry}
\geometry{a4paper,margin=0.85cm, bottom=0.5cm}
 %To reduce the height of the top margin uncomment: 
 \addtolength{\voffset}{+0.35cm}

\usepackage{hyperref} % Required for adding links	and customizing them
\definecolor{linkcolour}{rgb}{0,0.2,0.6} % Link color
\hypersetup{colorlinks,breaklinks,urlcolor=linkcolour,linkcolor=linkcolour} % Set link colors throughout the document

\usepackage{titlesec} % Used to customize the \section command
\titleformat{\section}{\Large\scshape\raggedright}{}{0em}{}[\titlerule] % Text formatting of sections
\titlespacing{\section}{0pt}{0pt}{0pt} % Spacing around sections

\usepackage{array}
\newcolumntype{L}[1]{>{\raggedright\let\newline\\\arraybackslash\hspace{0pt}}m{#1}}
\newcolumntype{C}[1]{>{\centering\let\newline\\\arraybackslash\hspace{0pt}}m{#1}}
\newcolumntype{R}[1]{>{\raggedleft\let\newline\\\arraybackslash\hspace{0pt}}m{#1}}


\begin{document}

\pagestyle{empty} % Removes page numbering

%\font\fb=''[cmr10]'' Change the font of the \LaTeX command under the skills section

%----------------------------------------------------------------------------------------
%	NAME AND CONTACT INFORMATION
%----------------------------------------------------------------------------------------
\par{\centering{\Huge \textsc{Srishti Samadder | 15CH10045}}\par} % Your name
\par{\centering\large {\textsc{Department of Chemical Engineering}}\par}\large
\par{\centering\large {\textsc{B.Tech Student at Indian Institute of Technology Kharagpur}}\par}\large
%\par{{\begin{center}Dual Degree, \emph{Computer Science and Engineering}\end{center}}}
\par{\centering\large {\textsc{A-312, MT Hall of Residence, IIT Kharagpur, West Bengal - 721302}}\par}\large
%\begin{flushleft}{\normalsize {\href{mailto:nareshmdu@gmail.com}{nareshmdu@gmail.com}}}\end{flushleft}
\hspace{3.5cm}\normalsize {\href{mailto:samadder.srishti@gmail.com}{samadder.srishti@gmail.com}}\hfill{Phone: +91-8697349681}\hspace{3.5cm}

%\section{Personal Data}
%
%\begin{tabular}{rl}
%\textsc{Place and Date of Birth:} & Canada  | 20 November 1987 \\
%\textsc{Address:} & 123 Broadway, City, State, Canada \\
%\textsc{Phone:} & +1 111 1112\\
%\textsc{email:} & \href{mailto:john@smith.com}{john@smith.com}
%\end{tabular}

%----------------------------------------------------------------------------------------
%	RESEARCH INTERESTS
%----------------------------------------------------------------------------------------

%\section{Research Interests}

%- Software Design\hfill
%- Image Processing\hspace{3.7cm} \\

%----------------------------------------------------------------------------------------
%	EDUCATION
%----------------------------------------------------------------------------------------

\section{Education}

\begin{tabular}{R{3.5cm}|p{14.5cm}}	
2015-2019 & B.Tech in \textsc{Chemical Engineering}\\
\textsc{(Expected)}&\textbf{Indian Institute of Technology Kharagpur}\hfill\textsc{Cgpa}: 7.8/10.0\\
%\hyperlink{grds}{\hfill | \footnotesize Detailed List of Exams}\\


%------------------------------------------------

2015& Class XII, \textsc{}\textsc{Central Board of Secondary Education (CBSE)} \\
%&110/110 \small\emph{Commerce Specialization},
&\normalsize\textbf{Bhavan's Gangabux Kanoria Vidyamandir, Saltlake, Kolkata} \hfill\textsc{Score}: 95.2\%\\
%\hyperlink{grds_usc}{\hfill| \footnotesize Detailed List of Exams}\\


%------------------------------------------------

2010 & Class X, \textsc{}\textsc{Central Board of Secondary Education (CBSE)} \\
&\normalsize\textbf{Bhavan's Gangabux Kanoria Vidyamandir, Saltlake, Kolkata} \hfill\textsc{Cgpa}: 10.0/10.0\\
%\hyperlink{grds_usc}{\hfill| \footnotesize Detailed List of Exams}\\

\end{tabular}

%----------------------------------------------------------------------------------------
%	Academic Projects
%----------------------------------------------------------------------------------------

\section{Projects Undertaken}

\begin{tabular}{R{3.5cm}|p{14.5cm}}
%\emph{Current} & 1\textsuperscript{st} year Analyst at \textsc{Lehman Brothers}, London \\
\textsc{Sep '16 to Current} & \textbf{Multimodal Deep Learning for Detection of Diabetic Retinopathy} \\
 & \textbf{Ph.D Incharge: }\textmd{Anirban Santara}\\
 
 & \textmd{ Working on using Convolutional Neural Networks for automatic diagnosis of Diabetic Retinopathy}\\
\multicolumn{2}{c}{} \\

\textsc{Jan '16 to Current} & \textbf{Swarm Robot} \textmd{(Autonomous mobile robots modelled as Swarms)} \\
 & \textbf{Group: }\textmd{Decentralized Terrain Exploration with Robot Swarms}, IIT Kharagpur\\
& \textbf{Guide: }\textmd{Professor Pallab Dasgupta} and \textmd{Professor Somesh Kumar}\\
%& \footnotesize{- Developing a robust autonomous mobile robot to participate in the annual AUVSI ROBOSUB held in San Diego, California.}\\
& \textmd{Worked on the code for \textbf{pose data communication} using basic TCP/IP protocol to update all robots' current pose data in a pose table built using ROS message modules with Gazebo. Also worked on the code for the movement of robots using path-planning algorithms like A-Star and APF and a dynamic leader selection using random message passing. Working on a Clustering Algorithm on swarm robots using Dynamic leader selection.}\\
\multicolumn{2}{c}{} \\

%\textsc{April 2015} & \textbf{Wikification}\\
%\textsc{March 2015} & \textbf{Guide: }\textmd{\href{http://cse.iitkgp.ac.in/~pawang/}{Professor Pawan Goyal}}\\
%& \footnotesize{- Worked on implementing a {\href{http://www.cs.sjtu.edu.cn/~kzhu/papers/wikification.pdf}{research paper}} to generate links to Wikipedia pages of terms in a page.}\\
%& \footnotesize{- The algorithm first identifies noun phrases and generates a Term-Sense mapping to disambiguate them. It then enriches the link co-occurrence matrix to improve the accuracy and then wikifies the article.}\\
%& \footnotesize{ -Got familiar with the NLTK chunker, which was used to identify noun phrases.}\\
%\multicolumn{2}{c}{} \\



%\textsc{Dec 2013} & \textbf{\href{http://www.robotix.in/tutorials/categ/auto/lfr}{Lane Follower Robot}} \\
%& \footnotesize{- Developed an autonomous line following robot using Atmel AVR microprocessor(Atmega16) in an IEEE certified workshop organized by Technology Robotix Scociety, IIT Kharagpur.}\\
\end{tabular}

%----------------------------------------------------------------------------------------
%	SKILLS 
%----------------------------------------------------------------------------------------

\section{Technical Skills}

\begin{tabular}{R{3.5cm}|p{14.5cm}}
\textsc{Languages} & {\itshape{Expert : }}C {\itshape{Intermediate : }} HTML, C++  {\itshape{Beginner : }}Python \\
%\multicolumn{2}{c}{} \\
\textsc{Operating Systems} &  Microsoft Windows, Linux (Ubuntu \& CentOS)\\
%\multicolumn{2}{c}{} \\
\textsc{Libraries and IDEs} & ROS, Gazebo, git, Eclipse, NetBeans, LaTeX, TensorFlow\\
\end{tabular}

%----------------------------------------------------------------------------------------
%	POSITIONS OF RESPONSIBILITY
%----------------------------------------------------------------------------------------

\section{Positions of Responsibility}

\begin{tabular}{R{3.5cm}|p{14.5cm}}
%& \footnotesize{- Managing the content and design team of the society.}\\
%& \footnotesize{- Writer in the English Team, and working as a senior editor for all English publications.}\\
%\textsc{Apr 2015} & \textbf{Secretary}, CodeClub, IIT Kharagpur \\
%& \footnotesize{- Part of the managing team, leading a group of 25 students.}\\
%& \footnotesize{- Conducted several events, including Microsoft code.fun.do and BITWISE, the Annual Departmental Fest of the Department of Computer Science and Engineering, alongside several fortnightly competitive coding competitions within the campus.}\\
%& \footnotesize{- Designed the software for updating the leaderboards in real-time during competitions as a part of BITWISE '15.}\\
\textsc{Aug '15 to Current} & \textbf{Actor}, \textsc{Encore, Technology Dramatics Society, IIT Kharagpur} \\
\textsc{Aug '15 to Jan '16} & \textbf{Actor}, \textsc{Druheen, Technology Dramatics Society, IIT Kharagpur} \\
\textsc{Aug '15 to Apr '16} & \textbf{Member}, \textsc{Debating Society, IIT Kharagpur} \\
%& \footnotesize{- Organized multiple workshops and events, primarily focused on Android Development, in association with Google.}\\
%& \footnotesize{- Conducted a workshop on the {\href{https://www.polymer-project.org/0.5/}{Polymer Project}}, which received high levels of}\\ & \footnotesize{participation.}\\
\end{tabular}


%----------------------------------------------------------------------------------------
%	COURSEWORK
%----------------------------------------------------------------------------------------

\section{Coursework
\hfill\small\textsc{(T)heory and (L)aboratory}}

{\hspace{0.5cm}- Programming and Data Structures (T/L)}
\\
{\hspace*{0.5cm} \itshape{Currently Studying:}}
\\
{\hspace*{0.5cm}- Algorithms (T/L)}\\
{\hspace*{0.5cm}- Machine Learning }
%----------------------------------------------------------------------------------------
%	ACHIEVEMENTS
%----------------------------------------------------------------------------------------

\section{Scholastic \& Extra-Curricular Achievements}

{\hspace{0.5cm}- Qualified JEE Mains 2015 out of 14 Lakh students}\\ \hspace*{0.5cm}{- Qualified JEE Advanced 2015 out of 5 Lakh students}\\
%{\hspace*{0.5cm}- Secured 4th Position in \textsc{TCS IT Wiz Quiz, 2009} out of more than 500 participating teams}\\
{\hspace*{0.5cm}- \textbf{Director} of \textsc{Security Council, BhavMUN 2012-13, BGKV, Kolkata}}\\
\hspace*{0.5cm}{- Selected to interview \textbf{Ms. Sunita Williams} from 12000 students in 50 schools all over Kolkata}\\
{\hspace*{0.5cm}- Secured \textbf{Bronze} in \textsc{\textbf{Rangmanch}} team events for \textbf{Spring Fest, 2015-16}}\\
{\hspace*{0.5cm}- One of 7 \textbf{Risk Latte Scholars} for the year 2016 chosen from college students all over India}

%----------------------------------------------------------------------------------------

%\newpage
%----------------------------------------------------------------------------------------
\end{document}